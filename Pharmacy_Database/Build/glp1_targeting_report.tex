\documentclass[11pt]{article}
\usepackage[margin=1in]{geometry}
\usepackage{xcolor}
\usepackage{enumitem}
\usepackage{booktabs}
\usepackage{longtable}
\usepackage{fancyhdr}
\usepackage{titlesec}
\usepackage{hyperref}
\usepackage{colortbl}
\usepackage{array}
\usepackage{tabularx}
\usepackage{multicol}
\usepackage{graphicx}
\usepackage[T1]{fontenc}
\usepackage{lmodern}
\usepackage[letterspace=150]{microtype}
\usepackage{tikz}
\usetikzlibrary{shapes.geometric}
\usepackage{footnote}

% Colors
\definecolor{navy}{HTML}{0F172A}
\definecolor{teal}{HTML}{0F766E}
\definecolor{teallight}{HTML}{14B8A6}
\definecolor{orange}{HTML}{F97316}
\definecolor{slate}{HTML}{64748B}
\definecolor{slatebg}{HTML}{F8FAFC}
\definecolor{lightgray}{HTML}{E2E8F0}
\definecolor{green}{HTML}{16A34A}
\definecolor{amber}{HTML}{D97706}
\definecolor{red}{HTML}{DC2626}
\definecolor{navydark}{HTML}{0A1120}
\definecolor{navymid}{HTML}{1E293B}
\definecolor{indigo}{HTML}{4338CA}
\definecolor{indigodeep}{HTML}{312E81}
\definecolor{indigolight}{HTML}{818CF8}
\definecolor{copper}{HTML}{C2724E}

% Header/Footer
\pagestyle{fancy}
\fancyhf{}
\fancyhead[L]{\small\color{slate}\textsc{GLP-1 Pharmacy Targeting --- Research Results}}
\fancyhead[R]{\small\color{slate}February 2026}
\fancyfoot[C]{\small\color{slate}\thepage}
\fancyfoot[R]{\small\color{slate}RetailMyMeds}
\renewcommand{\headrulewidth}{0.4pt}
\renewcommand{\footrulewidth}{0.4pt}

% Section styling
\titleformat{\section}{\Large\bfseries\color{navy}}{}{0em}{}[\color{orange}\titlerule]
\titleformat{\subsection}{\large\bfseries\color{teal}}{}{0em}{}
\titleformat{\subsubsection}{\normalsize\bfseries\color{navy}}{}{0em}{}

\hypersetup{colorlinks=true, linkcolor=teal, urlcolor=teal}

% Callout box
\newcommand{\callout}[2]{%
\noindent\fcolorbox{lightgray}{slatebg}{%
\begin{minipage}{\dimexpr\linewidth-2\fboxsep-2\fboxrule}
\smallskip
{\small\color{teal}\textsc{#1}}\par\smallskip
#2
\smallskip
\end{minipage}}}

\begin{document}

%% ═══════════════════════════════════════════════════════════════
%% CINEMATIC COVER PAGE
%% ═══════════════════════════════════════════════════════════════
\thispagestyle{empty}
\newgeometry{margin=0pt}
\pagecolor{navy}

\begin{tikzpicture}[remember picture, overlay]

  % --- Layer 1: Deep indigo-shifted navy base ---
  \fill[navy!92!indigodeep] (current page.south west) rectangle (current page.north east);

  % --- Layer 2: Primary indigo radial glow (lower-right) ---
  \shade[inner color=indigo!40!navy, outer color=transparent, opacity=0.45]
    ([xshift=4cm, yshift=-5cm]current page.center) circle (12cm);

  % --- Layer 3: Secondary copper radial glow (upper-left, warm) ---
  \shade[inner color=copper!30!navy, outer color=transparent, opacity=0.35]
    ([xshift=-5cm, yshift=7cm]current page.center) circle (10cm);

  % --- Layer 4: Indigo bloom behind title area ---
  \shade[inner color=indigo!25!navy, outer color=transparent, opacity=0.30]
    ([xshift=0cm, yshift=4cm]current page.center) circle (6cm);

  % --- Layer 5: Warm copper underpaint ---
  \shade[inner color=copper!15!navy, outer color=transparent, opacity=0.20]
    ([xshift=2cm, yshift=-2cm]current page.center) circle (8cm);

  % --- Concentric ellipses (analytical/scientific feel) ---
  \foreach \r/\a in {3.0/0.16, 4.5/0.13, 6.2/0.10, 8.0/0.08, 10.0/0.06, 12.5/0.04} {
    \draw[indigolight!70!white, opacity=\a, line width=0.5pt, rotate around={-15:([xshift=2.5cm,yshift=-2.5cm]current page.center)}]
      ([xshift=2.5cm,yshift=-2.5cm]current page.center) ellipse ({\r cm} and {\r*0.58 cm});
  }

  % --- Second set of ellipses (offset, copper, rotated differently) ---
  \foreach \r/\a in {3.8/0.10, 5.5/0.07, 7.5/0.05, 9.5/0.04} {
    \draw[copper!55!white, opacity=\a, line width=0.4pt, rotate around={22:([xshift=-3.5cm,yshift=3.5cm]current page.center)}]
      ([xshift=-3.5cm,yshift=3.5cm]current page.center) ellipse ({\r cm} and {\r*0.52 cm});
  }

  % --- Bezier flow curves ---
  \draw[indigolight!50!white, opacity=0.12, line width=0.6pt]
    ([xshift=-3cm,yshift=-10cm]current page.center) ..
    controls ([xshift=5cm,yshift=-3cm]current page.center) and ([xshift=-4cm,yshift=5cm]current page.center) ..
    ([xshift=7cm,yshift=11cm]current page.center);
  \draw[copper!40!white, opacity=0.09, line width=0.5pt]
    ([xshift=-9cm,yshift=-5cm]current page.center) ..
    controls ([xshift=-1cm,yshift=3cm]current page.center) and ([xshift=4cm,yshift=6cm]current page.center) ..
    ([xshift=10cm,yshift=4cm]current page.center);
  \draw[indigolight!35!white, opacity=0.07, line width=0.4pt]
    ([xshift=8cm,yshift=-11cm]current page.center) ..
    controls ([xshift=2cm,yshift=-1cm]current page.center) and ([xshift=-5cm,yshift=4cm]current page.center) ..
    ([xshift=-8cm,yshift=10cm]current page.center);

  % --- Corner brackets (top-left) ---
  \draw[indigolight!60!white, opacity=0.25, line width=0.7pt]
    ([xshift=1.2cm, yshift=-1.2cm]current page.north west) -- ++(0, -2.2cm);
  \draw[indigolight!60!white, opacity=0.25, line width=0.7pt]
    ([xshift=1.2cm, yshift=-1.2cm]current page.north west) -- ++(2.2cm, 0);

  % --- Corner brackets (bottom-right) ---
  \draw[copper!50!white, opacity=0.20, line width=0.7pt]
    ([xshift=-1.2cm, yshift=1.2cm]current page.south east) -- ++(0, 2.2cm);
  \draw[copper!50!white, opacity=0.20, line width=0.7pt]
    ([xshift=-1.2cm, yshift=1.2cm]current page.south east) -- ++(-2.2cm, 0);

  % --- Horizontal measurement rules ---
  \draw[white, opacity=0.08, line width=0.3pt]
    ([xshift=1.5cm, yshift=-4.5cm]current page.north west) -- ([xshift=-1.5cm, yshift=-4.5cm]current page.north east);
  \draw[white, opacity=0.05, line width=0.3pt]
    ([xshift=1.5cm, yshift=3.5cm]current page.south west) -- ([xshift=-1.5cm, yshift=3.5cm]current page.south east);

  % --- Diamond accents ---
  \node[diamond, draw=indigolight!50!white, opacity=0.18, line width=0.5pt, minimum size=7pt, inner sep=0pt]
    at ([xshift=7cm,yshift=-4cm]current page.center) {};
  \node[diamond, draw=copper!40!white, opacity=0.14, line width=0.5pt, minimum size=5pt, inner sep=0pt]
    at ([xshift=-6cm,yshift=6cm]current page.center) {};
  \node[diamond, draw=indigolight!35!white, opacity=0.10, line width=0.4pt, minimum size=4pt, inner sep=0pt]
    at ([xshift=-7.5cm,yshift=-6cm]current page.center) {};

  % --- Edge vignette ---
  \shade[top color=black!50, bottom color=transparent, opacity=0.20]
    (current page.north west) rectangle ([yshift=-3cm]current page.north east);
  \shade[bottom color=black!50, top color=transparent, opacity=0.25]
    ([yshift=3cm]current page.south west) rectangle (current page.south east);
  \shade[left color=black!40, right color=transparent, opacity=0.15]
    (current page.north west) rectangle ([xshift=2.5cm]current page.south west);
  \shade[right color=black!40, left color=transparent, opacity=0.15]
    ([xshift=-2.5cm]current page.north east) rectangle (current page.south east);

  % --- Charioteer logo --- ghosted ---
  \node[opacity=0.06] at ([yshift=5.8cm]current page.south)
    {\includegraphics[width=4.5cm]{assets/charioteer-white.pdf}};

\end{tikzpicture}

\vspace*{3.5cm}

\begin{center}

{\fontsize{9}{11}\selectfont\textcolor{indigolight!70!white}{\textls[250]{\MakeUppercase{Research Results}}}}

\vspace{0.6cm}

{\fontsize{36}{42}\selectfont\bfseries\color{white}GLP-1 Pharmacy}

\vspace{0.15cm}

{\fontsize{36}{42}\selectfont\bfseries\color{white}Targeting}

\vspace{0.4cm}

{\color{indigo!70!white}\rule{6cm}{0.8pt}}

\vspace{0.5cm}

{\fontsize{13}{17}\selectfont\color{white!75}GLP-1 Economics\hspace{1em}|\hspace{1em}Pharmacy Scoring\hspace{1em}|\hspace{1em}Outreach Priority}

\vspace{0.4cm}

{\fontsize{12}{15}\selectfont\color{copper}\textsc{RETAILMYMEDS}}

\vspace{1.8cm}

{\fontsize{10}{13}\selectfont
\textcolor{white!50}{\textls[100]{\MakeUppercase{Prepared For}}}~~\textcolor{white!70}{\textbf{Arica Collins, PharmD}~~\&~~\textbf{Kevin McCarron}}\par
\vspace{8pt}
\textcolor{white!50}{\textls[100]{\MakeUppercase{Prepared By}}}~~\textcolor{white!70}{\textbf{Matthew Scott}}\par
\vspace{4pt}
\textcolor{white!40}{February 2026~~|~~v1.0}
}

\end{center}

\vfill

\begin{center}
{\fontsize{8}{10}\selectfont\textcolor{white!35}{\textls[150]{\MakeUppercase{Project Lavos LLC}}}\quad\textcolor{indigo!50!white}{|}\quad\textcolor{white!35}{Louisville, KY}}
\end{center}

\vspace{0.8cm}

\restoregeometry
\nopagecolor

\newpage

%% ─────────────────────────────────────────────────────────────
\section{What This Is}

I ran research against publicly available federal datasets to answer a specific question: \textbf{which of the 41,775 qualified independent pharmacies in the database are the strongest targets for RetailMyMeds, and why?}

The result is a single CSV (\texttt{pharmacies\_glp1\_targeting.csv}) that scores and ranks every pharmacy by estimated GLP-1 financial exposure, area-level health data, and outreach priority. Every number traces to a named federal source.

The underlying data is measured --- federal agency datasets, not survey self-reports. The financial projections layered on top are \textit{estimates} built from that data using NCPA's published loss-per-fill economics.\textsuperscript{1,2} Where a number is measured vs.\ estimated is noted throughout.

%% ─────────────────────────────────────────────────────────────
\section{Data Sources}

\begin{tabularx}{\textwidth}{>{\bfseries}l X l}
\toprule
Source & What It Provides & Agency \\
\midrule
Medicare Part D\textsuperscript{3} & State GLP-1 claims + drug cost (2023) & CMS \\
Medicaid SDUD\textsuperscript{4} & State GLP-1 prescriptions + reimbursement (2024) & CMS \\
CDC PLACES 2025\textsuperscript{5} & ZIP-level diabetes, obesity, uninsured rates & CDC \\
Census ACS 2023\textsuperscript{6} & ZIP-level \% 65+, median income, population & Census Bureau \\
HRSA HPSA 2026\textsuperscript{7} & Primary care shortage area designations & HRSA \\
NPPES NPI Registry\textsuperscript{8} & Pharmacy identity, status, owner & CMS \\
\bottomrule
\end{tabularx}

\vspace{0.15in}

Loss-per-fill economics (\$37--\$42/fill) come from NCPA survey data\textsuperscript{1,2} and per-drug WAC-minus-reimbursement calculations from the GLP-1 Value Proposition. The \$275/month subscription price and 5\% moderate routing rate come from the RetailMyMeds GLP-1 Value Proposition.

%% ─────────────────────────────────────────────────────────────
\section{How the Scoring Works}

Each pharmacy receives a \textbf{final score} (0--100) from three weighted components:

\begin{tabularx}{\textwidth}{l c X}
\toprule
\textbf{Component} & \textbf{Weight} & \textbf{What It Measures} \\
\midrule
Opportunity Score & 45\% & Area disease burden (diabetes + obesity), Medicare density (\% 65+), state GLP-1 market intensity, HPSA underserved status \\
Financial Impact & 30\% & Magnitude of estimated monthly GLP-1 loss (\$) \\
Urgency & 25\% & GLP-1 fill volume, disease severity, MFP exposure, shortage area acuity \\
\bottomrule
\end{tabularx}

\vspace{0.15in}

\textbf{Why these weights.} Opportunity captures whether the pharmacy is \textit{in the right area} --- high disease burden means more GLP-1 demand, more 65+ means more Part D exposure, HPSA means less competition. Financial impact captures how much the pharmacy is \textit{bleeding today}. Urgency captures how \textit{acutely} they need a solution. A pharmacy can be in a perfect area (high opportunity) but with moderate volume (lower financial impact) --- or vice versa. The weighted blend identifies pharmacies where all three align.

\vspace{0.15in}

\callout{Fill Estimate Methodology}{%
Monthly GLP-1 fills per pharmacy are estimated by indexing each state's CMS-measured claims volume\textsuperscript{3,4} against the NCPA national average of 394 GLP-1 fills/month,\textsuperscript{9} then adjusting $\pm$15\% based on the pharmacy's ZIP-level disease burden\textsuperscript{5} and Medicare density.\textsuperscript{6} The resulting range (197--650 fills/month) aligns with the 80--580 range observed in the 12-pharmacy portfolio analysis.
}

%% ─────────────────────────────────────────────────────────────
\section{Results}

\subsection{Grade Distribution}

\begin{tabularx}{\textwidth}{c c c X}
\toprule
\textbf{Grade} & \textbf{Count} & \textbf{\% of Total} & \textbf{Classification} \\
\midrule
\cellcolor{green!15}\textcolor{green}{\textbf{A}} & 5,118 & 12.3\% & Strong Fit --- high area need + high financial exposure + high urgency \\
\cellcolor{teallight!15}\textcolor{teal}{\textbf{B}} & 14,775 & 35.4\% & Good Fit --- strong on two of three dimensions \\
\cellcolor{amber!15}\textcolor{amber}{\textbf{C}} & 18,352 & 43.9\% & Conditional --- moderate opportunity, worth monitoring \\
\cellcolor{red!15}\textcolor{red}{\textbf{D}} & 3,530 & 8.5\% & Low priority --- low volume or low area need \\
\bottomrule
\end{tabularx}

\subsection{Outreach Priority}

\begin{tabularx}{\textwidth}{l c c X}
\toprule
\textbf{Priority} & \textbf{Count} & \textbf{\%} & \textbf{Action} \\
\midrule
\textcolor{green}{\textbf{Immediate Outreach}} & 11,190 & 26.8\% & Direct outreach --- final score $\geq$ 80 \\
\textcolor{teal}{\textbf{Nurture}} & 10,949 & 26.2\% & Drip campaign, conference touchpoints --- score 70--79 \\
\textcolor{amber}{\textbf{Conditional}} & 16,106 & 38.6\% & Watch list --- score 55--69, may convert as MFP cycles expand \\
\textcolor{red}{\textbf{Deprioritize}} & 3,530 & 8.5\% & Score below 55 --- do not spend time here \\
\bottomrule
\end{tabularx}

\subsection{Conversation Segments}

Each pharmacy is tagged with the \textbf{pain point that should lead the conversation}:

\begin{tabularx}{\textwidth}{l c X}
\toprule
\textbf{Segment} & \textbf{Count} & \textbf{Lead Message} \\
\midrule
GLP-1 Loss & 30,808 (73.7\%) & ``You're losing \$37--\$42 on every GLP-1 fill. RetailMyMeds routes those prescriptions so you stop absorbing the loss.'' \\
DIR Fee Squeeze & 7,424 (17.8\%) & ``Your area is medically underserved and your margins are the thinnest in the country. Routing below-cost prescriptions is how you survive.'' \\
MFP Cash Flow & 3,543 (8.5\%) & ``MFP is creating a \$10,838/week cash flow gap. RetailMyMeds' scheduling intelligence addresses the timing directly.'' \\
\bottomrule
\end{tabularx}

%% ─────────────────────────────────────────────────────────────
\section{Financial Model}

Every pharmacy in the CSV includes a per-pharmacy financial projection. ``ROI multiple'' = estimated annual net savings / \$3,300 annual subscription cost, assuming 5\% of GLP-1 fills are routed at the mid-range loss of \$39.50/fill. This is a projected return on the subscription, not a guaranteed outcome --- actual results depend on the pharmacy's real fill volume and PBM contracts.

\callout{Immediate Outreach Pharmacies --- Average Profile}{%
\begin{tabular}{rl}
Estimated monthly GLP-1 fills: & \textbf{$\sim$484} \\
Estimated monthly GLP-1 loss (mid): & \textbf{$\sim$\$19K} (\$10K--\$26K range) \\
At 5\% routing rate ($\sim$24 fills routed): & \textbf{$\sim$\$680/month net savings} after \$275 subscription \\
Estimated ROI multiple: & \textbf{2.5x} (range: 0.9x--3.6x) \\
Breakeven: & \textbf{8 fills/month} routed ($\sim$2\% of GLP-1 volume) \\
\end{tabular}

\vspace{0.1in}

{\small\color{slate}\textit{These are estimates, not pharmacy-reported numbers. The national average pharmacy (394 fills/month at \$39.50/fill mid-range loss) yields \$503/month net savings and 1.8x ROI per the GLP-1 Value Proposition.\textsuperscript{9} Immediate Outreach pharmacies score higher because they are in areas with above-average disease burden and Medicare density, which the model uses to estimate higher fill volume.}}
}

\vspace{0.15in}

The CSV includes three loss scenarios per pharmacy:

\begin{tabular}{l l l}
\toprule
\textbf{Scenario} & \textbf{Loss/Fill} & \textbf{Source} \\
\midrule
Conservative & \$37/fill & Per-drug WAC/reimbursement spread\textsuperscript{1,9} \\
Mid-range & \$39.50/fill & Midpoint \\
High & \$42/fill & NCPA pharmacist survey\textsuperscript{2} \\
\bottomrule
\end{tabular}

%% ─────────────────────────────────────────────────────────────
\section{Top 10 States by Immediate Outreach Volume}

\begin{tabularx}{\textwidth}{c l c r r}
\toprule
\textbf{Rank} & \textbf{State} & \textbf{Imm. Outreach} & \textbf{Est.\ Avg Loss/Mo} & \textbf{Est.\ ROI} \\
\midrule
1 & California & 2,070 & $\sim$\$20K & 2.6x \\
2 & North Carolina & 883 & $\sim$\$18K & 2.3x \\
3 & Ohio & 842 & $\sim$\$25K & 3.6x \\
4 & \textbf{Louisiana} & \textbf{759} & $\sim$\textbf{\$20K} & \textbf{2.5x} \\
5 & Illinois & 671 & $\sim$\$17K & 2.1x \\
6 & Tennessee & 574 & $\sim$\$15K & 1.8x \\
7 & Missouri & 437 & $\sim$\$15K & 1.8x \\
8 & South Carolina & 395 & $\sim$\$18K & 2.2x \\
9 & West Virginia & 356 & $\sim$\$19K & 2.4x \\
10 & Wisconsin & 354 & $\sim$\$21K & 2.8x \\
\bottomrule
\end{tabularx}

\vspace{0.1in}

{\small Louisiana is highlighted as Arica's home market. Ohio and Wisconsin show the highest per-pharmacy loss among the top 10, making them strong expansion targets.}

%% ─────────────────────────────────────────────────────────────
\section{What's in the CSV}

\texttt{pharmacies\_glp1\_targeting.csv} contains \textbf{36 columns} per pharmacy:

\begin{multicols}{2}
\subsubsection*{Identity (9 fields)}
\begin{itemize}[nosep,leftmargin=1em]
\item NPI, name, city, state, ZIP, phone
\item NPPES status, owner name, owner title
\end{itemize}

\subsubsection*{Area Health Context (7 fields)}
\begin{itemize}[nosep,leftmargin=1em]
\item ZIP diabetes \%, obesity \%, \% 65+
\item Median income, population
\item HPSA designated (yes/no), HPSA score
\end{itemize}

\subsubsection*{State GLP-1 Economics (2 fields)}
\begin{itemize}[nosep,leftmargin=1em]
\item Govt GLP-1 claims per pharmacy
\item Govt GLP-1 cost per pharmacy
\end{itemize}

\subsubsection*{Pharmacy Financial Estimates (5 fields)}
\begin{itemize}[nosep,leftmargin=1em]
\item Est.\ monthly GLP-1 fills
\item Monthly loss (low / mid / high)
\item Annual loss
\end{itemize}

\subsubsection*{ROI Model (7 fields)}
\begin{itemize}[nosep,leftmargin=1em]
\item Fills routed at 5\%, monthly savings
\item Net monthly after \$275 subscription
\item Annual net savings, ROI multiple
\item Breakeven fills, \% volume for breakeven
\end{itemize}

\subsubsection*{Targeting (6 fields)}
\begin{itemize}[nosep,leftmargin=1em]
\item Segment (GLP-1 / MFP / DIR)
\item Urgency score, opportunity score
\item Final score, grade, outreach priority
\end{itemize}
\end{multicols}

%% ─────────────────────────────────────────────────────────────
\section{What This Means}

The 41,775 pharmacies in the qualified database now have \textbf{measured, source-cited targeting data}. For Arica's outreach:

\begin{enumerate}[leftmargin=1.5em]
\item \textbf{11,190 pharmacies are flagged Immediate Outreach.} These are independents in high-disease-burden, high-Medicare-density areas with the highest estimated GLP-1 losses. They are the pharmacies most likely to feel the pain RetailMyMeds solves.

\item \textbf{Every pharmacy has a conversation segment.} The CSV tells you whether to lead with GLP-1 losses, MFP cash flow, or DIR fee pressure --- based on that pharmacy's area characteristics, not a guess.

\item \textbf{The financial model is per-pharmacy, not per-state.} Each row shows estimated fills, estimated losses, projected savings at \$275/month, and estimated ROI. This applies the same math from the GLP-1 Value Proposition to each pharmacy's area context. The base case (394 fills, \$37/fill, 5\% routing) yields \$465/month net --- higher-scoring pharmacies project higher because their area data suggests above-average GLP-1 volume.

\item \textbf{Louisiana has 759 Immediate Outreach pharmacies} with an average estimated loss of $\sim$\$20K/month and 2.5x estimated ROI. This is Arica's home market, supported by the WVIPA relationship and association expansion strategy.

\item \textbf{This data feeds the qualification system Kevin is building on the site.} The targeting database identifies which pharmacies to reach. When those pharmacies engage --- through Arica's outreach or through the website --- the qualification form scores them using the same dimensions (financial exposure, operational readiness, market urgency) and delivers a personalized scorecard. The database is the prospect list; the website is the conversion point.
\end{enumerate}

\vspace{0.15in}

\callout{Concrete Next Step}{%
In the CSV, filter \texttt{outreach\_priority = ``Immediate Outreach''} and \texttt{state = ``LA''} to get Arica's first 759 call targets, sorted by \texttt{final\_score} descending. The \texttt{segment} column tells you which pain point to lead with for each pharmacy.
}

\vspace{0.2in}

\callout{Honest Framing}{%
\textbf{What is measured:} NPI/pharmacy identity (NPPES\textsuperscript{8}), ZIP-level disease prevalence (CDC PLACES\textsuperscript{5}), \% 65+ and income (Census ACS\textsuperscript{6}), HPSA designations (HRSA\textsuperscript{7}), state-level GLP-1 claims and costs (CMS Medicare Part D\textsuperscript{3}, Medicaid SDUD\textsuperscript{4}).

\textbf{What is estimated:} Per-pharmacy GLP-1 fill volume (derived from state CMS data indexed to the NCPA 394/month average,\textsuperscript{9} adjusted $\pm$15\% by area health burden), monthly loss (fills $\times$ \$37--\$42 NCPA loss/fill\textsuperscript{1,2}), savings (assumes 5\% routing rate from Value Proposition), and ROI (savings / \$3,300 annual subscription).

\textbf{Key assumptions in the estimation chain:} Government payers represent $\sim$55\% of GLP-1 volume; state index is capped at 1.5x national average; area disease burden adjusts fills $\pm$15\%; 5\% routing rate is the moderate scenario, not guaranteed.

Actual conversion depends on the pharmacy's specific PBM contracts, real fill volume, and willingness to adopt.
}

%% ─────────────────────────────────────────────────────────────
\section{Sources}

\begin{enumerate}[leftmargin=2em, label={\textsuperscript{\arabic*}}, nosep]

\item NCPA, ``Local pharmacies say they're struggling to afford GLP-1s,'' November 1, 2023. Reports \$37+ average losses per 30-day supply and 88\% of pharmacies considering discontinuing GLP-1 dispensing. Per-drug WAC/reimbursement spread confirms \$20--\$42 loss range across Ozempic, Wegovy, Mounjaro, and Zepbound.\\
\url{https://ncpa.org/newsroom/qam/2023/11/01/local-pharmacies-say-theyre-struggling-afford-glp-1s}

\item NCPA pharmacist survey, 2023. 95\% of respondents reported being paid an average of \$42 below acquisition cost on GLP-1 fills. 73\% reported turning away patients due to losses.\\
\url{https://ncpa.org/newsroom/qam/2023/11/01/local-pharmacies-say-theyre-struggling-afford-glp-1s}

\item CMS, Medicare Part D Spending by Drug, 2023. State-level GLP-1 claims volume and gross drug cost.\\
\url{https://data.cms.gov/summary-statistics-on-use-and-payments/medicare-medicaid-spending-by-drug/medicare-part-d-spending-by-drug}

\item CMS, Medicaid State Drug Utilization Data (SDUD), 2024. State-level GLP-1 prescriptions and reimbursement amounts.\\
\url{https://www.medicaid.gov/medicaid/prescription-drugs/state-drug-utilization-data}

\item CDC, PLACES: Local Data for Better Health, 2025 Release. ZIP-level model-based estimates for diabetes prevalence, obesity prevalence, and other chronic disease indicators.\\
\url{https://www.cdc.gov/places/index.html}

\item U.S.\ Census Bureau, American Community Survey 5-Year Estimates, 2019--2023. ZIP-level \% age 65+, median household income, and population.\\
\url{https://data.census.gov/all?y=2023&d=ACS+5-Year+Estimates+Detailed+Tables}

\item HRSA, Health Professional Shortage Areas (HPSA), current as of January 2026. Primary care shortage area designations and HPSA scores.\\
\url{https://data.hrsa.gov/topics/health-workforce/shortage-areas}

\item CMS, National Plan and Provider Enumeration System (NPPES) NPI Registry. Pharmacy identity, taxonomy, status, and owner information.\\
\url{https://download.cms.gov/nppes/NPI_Files.html}

\item RetailMyMeds GLP-1 Routing Value Proposition, February 2026 (v1.0). 394 fills/month national average derived from 67,601 annual Rx at 7\% GLP-1 penetration (NCPA Digest 2024). \$275/month pricing, 5\% moderate routing rate, breakeven analysis.

\end{enumerate}

\end{document}
